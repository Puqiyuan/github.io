% Created 2019-01-11 Fri 18:23
% Intended LaTeX compiler: pdflatex
\documentclass[11pt]{article}
\usepackage[utf8]{inputenc}
\usepackage[T1]{fontenc}
\usepackage{graphicx}
\usepackage{grffile}
\usepackage{longtable}
\usepackage{wrapfig}
\usepackage{rotating}
\usepackage[normalem]{ulem}
\usepackage{amsmath}
\usepackage{textcomp}
\usepackage{amssymb}
\usepackage{capt-of}
\usepackage{hyperref}
\author{Cauchy}
\date{\today}
\title{管道与重定向}
\hypersetup{
 pdfauthor={Cauchy},
 pdftitle={管道与重定向},
 pdfkeywords={},
 pdfsubject={},
 pdfcreator={Emacs 26.1 (Org mode 9.1.14)}, 
 pdflang={English}}
\begin{document}

\maketitle
\tableofcontents

管道与重定向是Linux系统中基本的概念,本文结合一些实例来验证Linux系统中这两个基础概念。

Linux系统中的每一个进程都有关联的三个文件, \texttt{STDIN} , \texttt{STDOUT} , \texttt{STDERR} , 内核为进程
打开这三个文件,内核会保证这三个文件的描述符分别为0,1,2。所以用数字来引用这三个文件是安
全的。

符号 \texttt{<} , \texttt{>} , \texttt{>>} 会对一个命令的输出与输入进行重定向。符号 \texttt{<} 使得命令的输入可以从
一个已存在的文件中读入。而 \texttt{>} 与 \texttt{>>} 重定向 \texttt{STDOUT} , \texttt{>} 会替代原文件中的内容,而
\texttt{>>} 只会附接在原文件尾。下面这个命令:

\begin{verbatim}
echo "This is a test message." > /tmp/mymessage
\end{verbatim}

将会向 \texttt{/tmp/mymessage} 中写入一行,如果这个文件不存在就先创建它。

而下面的例子展示了一个命令如何接受来自已存在文件的输入:

\begin{verbatim}
mail -s "Mail test" johndoe < /tmp/mymessage
\end{verbatim}

如你所见, \texttt{mail} 命令接受来自文件 \texttt{/tmp/mymessage} 的信息作为要发送给johndoe邮件的内容。

\textbf{符号>\&用来将STDOUT和STDERR重定向到相同的地方。而若只重定向STDERR,用符号2>。}

下面的例子解释了为什么有时候需要将STDOUT和STDERR重定向到不同的地方。

\begin{verbatim}
find / -name core
\end{verbatim}

通常情况下它的执行会引起大量的“权限受限”错误信息,这么多的信息无益于提取有用的信息。这时候
符号2>就派上用场了,它可以将错误信息重定向到其它地方:

\begin{verbatim}
find / -name core 2> /dev/null
\end{verbatim}

当然,对于有用信息的提取也可以放到一个文件中以便以后来查看:

\begin{verbatim}
find / -name core > /tmp/corefiles 2> /dev/null
\end{verbatim}

向 \texttt{/dev/null} 文件中写入信息,实际上就是丢弃这些信息。

\textbf{如果要把一个命令的STDOUT作为另一个命令的STDIN,这就要用到符号|,这就是管道。} 下面几个例
 子展示了这种用法:

\begin{verbatim}
ps -ef | grep httpd
cut -d: -f7 < /etc/passwd | sort -u
\end{verbatim}

第一个命令中的ps -ef将运行结果输出到grep,而grep的运行结果不再管道到别的地方,就输出到终
端窗口。而第二个命令用法类似。

使用符号\&\&连接两个命令,则只有第一个命令成功时,第二个命令才会执行。例如:

\begin{verbatim}
lpr /tmp/t2 && rm /tmp/t2
\end{verbatim}
\end{document}